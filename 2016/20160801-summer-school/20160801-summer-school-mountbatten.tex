% Footnotes used inline in the text.
\newcommand{\manualhandlingfootnote}{\footnote{\url{http://www.hse.gov.uk/pubns/indg143.pdf}}}
\newcommand{\chargingfootnote}{\footnote{\url{https://www.studentrobotics.org/docs/kit/batteries/imax_b6_charger\#ChargingChecklist}}}
\newcommand{\estatesfacilitiesfootnote}{\footnote{Estates and Facilities: \url{http://www.southampton.ac.uk/estates/}}}
\newcommand{\esofootnote}{\footnote{Equipment and Safety Officer; for the duration of this event, this role will be filled by Kier Davis (\email{me@kierdavis.com}).}}


\newcommand{\groupname}{Student Robotics (Southampton)}
\newcommand{\assessorname}{Kier Davis}
\newcommand{\assessoremail}{me@kierdavis.com}
\newcommand{\assessmentdate}{June 21, 2016}


\newcommand{\activityname}{Summer School (use of Mountbatten building)}
\newcommand{\activitydate}{August 1 to August 5, 2016}
\newcommand{\activitytime}{All day}
\newcommand{\activitylocation}{Room 53/4025 (Mountbatten building)}
\newcommand{\activitysummary}{
    The summer school is run by the ECS department with assistance from the
    Student Robotics organisation. The participants are invited to design,
    build and test an autonomous robot to perform a complex task. In addition to
    this main event, there are a number of taught laboratory sessions. Members
    of Student Robotics will mentor and provide assistance to the participants,
    as well as set up the arena for the competition and lead the laboratory
    sessions.

    This risk assessment only covers the use of the room 53/4025, which is used
    for access to the storage cupboard.
}


\newcommand{\references}{
    \reference{Guidance from the Health and Safety Executive, including manual
    handling procedures. \\
    \url{http://www.hse.gov.uk/risk/index.htm}}

    \reference{H\&S guidance from the Union Southampton website. \\
    \url{https://www.unionsouthampton.org/groups/admin/howto/protection}}

    \reference{Risk assessments prepared for events run by Student Robotics in 2015. \\
    \url{https://github.com/srobo-southampton/risk-assessments/tree/master/old}}

    \reference{Lithium polymer battery charging procedure, available on the
    Student Robotics website. \\
    \url{https://www.studentrobotics.org/docs/kit/batteries/imax_b6_charger\#ChargingChecklist}}
}


\newcommand{\risks}{
    \risk
        {Manual handling of heavy objects}
        {Mentors could experience minor injury or back pains resulting from
         improper lifting methods.}
        {\item The HSE manual handling guidelines\manualhandlingfootnote are to
         be followed for all tasks involving heavy lifting.}
        {\item No further action required.}
        {1} % Likelihood (/3)
        {2} % Impact (/3)

    \risk
        {Obstacles on the floor, such as bags, boxes or trailing cables}
        {Mentors may suffer injury as a result of tripping.}
        {\item Cables (such as laptop power supplies) will be routed underneath
         desks wherever possible.
         \item Cables that cannot be routed under desks will be clearly marked
         to increase their visibility.
         \item Bags, boxes and other items that are potential trip hazards will
         be stacked neatly by the walls whenever possible.}
        {\item No further action required.}
        {2} % Likelihood (/3)
        {1} % Impact (/3)

    \risk
        {Lithium polymer batteries}
        {LiPo batteries can ignite if damaged or misused, resulting in
         smoke/fire.}
        {\item Batteries will be routinely inspected for signs of damage or
         swelling, and set aside for safe disposal if necessary.
         \item Batteries are only to be charged by trained mentors. The battery
         charging procedure\chargingfootnote is to be followed at all times.}
        {\item No further action required.}
        {1} % Likelihood (/3)
        {2} % Impact (/3)
}


\newcommand{\postrisks}{
    % \newpage

    \subsection*{Risk of fire}

    To minimise the risk of fire resulting from this activity, food and drink
    will not be allowed near electrical equipment, and naked flames will be
    prohibited. The risk of fire occurring elsewhere in the building is
    controlled primarily by the building operator\estatesfacilitiesfootnote.
    The ESO\esofootnote will ensure that all people present are informed of the
    locations of the exits and that no fire drills are expected to take place.
}


% ================================================
% ============== ASSESSMENT DETAILS ==============
% ================================================

% The name of the student society.
\newcommand{\groupname}{Student Robotics (Southampton)}

% The name and email address of the person preparing this risk assessment.
\newcommand{\assessorname}{Kier Davis}
\newcommand{\assessoremail}{me@kierdavis.com}

% The date on which this risk assessment was completed.
\newcommand{\assessmentdate}{Jan 1, 2000}


% ================================================
% =============== ACTIVITY DETAILS ===============
% ================================================

% The name of the activity (either an event, or "General Activities").
\newcommand{\activityname}{World Robot Appreciation Day}

% The date(s) and time(s) of the activity.
% For "General Activities", these may be "Wednesday afternoons" and
% "17:00--23:00", for example.
\newcommand{\activitydate}{Jan 1, 2000}
\newcommand{\activitytime}{All day}

% The location the activity takes place in.
% The seminar room in Mountbatten (where the broomcupboard is) is 53/4025.
\newcommand{\activitylocation}{Room 53/4025 (Mountbatten building)}

% A summary of the activity. It should describe:
%   - what groups of people are attending, and approximate numbers of people in
%     each group
%   - what tasks will be done
%   - what equipment will be involved
%   - any other relevant information
\newcommand{\activitysummary}{
    % ...
}


% ================================================
% ================== REFERENCES ==================
% ================================================

% A list of external sources of information that were referred to when preparing
% this document. A couple of examples are given here.
\newcommand{\references}{
    \reference{Guidance from the Health and Safety Executive. \\
    \url{http://www.hse.gov.uk/risk/index.htm}}
    
    \reference{H\&S guidance from the Union Southampton website. \\
    \url{https://www.unionsouthampton.org/groups/admin/howto/protection}}
    
    % ...
}


% ================================================
% ==================== RISKS =====================
% ================================================

% A list of hazards, their control measures and their risk levels.
% Likelihood levels:
%   1 - May only occur in exceptional circumstances
%   2 - Might occur in some circumstances
%   3 - Will likely occur in most circumstances
% Impact levels:
%   1 - Superficial or minor injury. Can usually be handled by local first aid
%       procedures.
%   2 - Serious injury, possibly resulting in hospitalisation for up to thre
%       days. Complete recovery/rehabilitation could take several months.
%   3 - Major or fatal injury. Requires extensive medical treatment, including
%       at least three days in hospital.
\newcommand{\risks}{
    \risk
        {<a hazardous task or object>}
        {<who may be affected by this hazard, and in what ways>}
        {\item <a control measure already in place>
         \item <another control measure already in place>
         \item ...}
        {\item <an additional control measure that needs to be put in place>
         \item <another additional control measure that needs to be put in place>
         \item ...}
        {<likelihood level>}
        {<impact level>}
    
    % ...
}


% Ensure that this points to riskassessment.tex, which is found in the
% repository root.
\NeedsTeXFormat{LaTeX2e}
\ProvidesClass{riskassessment}

% Risk Assessments are landscape and A4
\LoadClass[a4paper,landscape]{article}

% Add packages
\RequirePackage{titlesec,array,calc,footnote}
\RequirePackage{hyperref,intcalc,longtable}
\RequirePackage{multicol}
\RequirePackage[margin=2cm]{geometry}

% Allow footnotes inside tables
\makesavenoteenv{tabular*}

% === CONFIGURE PAGE LAYOUT ===
\setlength{\parskip}{5pt}       % 5pt spacing between paragraphs
\setlength{\parindent}{0pt}     % No paragraph indentation
\setlength{\columnsep}{1.5cm}   % 1.5cm gutter between text columns


% === DEFINE COMMANDS ===

% Show an email address
\newcommand{\email}[1]{\href{mailto:#1}{#1}}

% Set the activity name as a title
\newcommand{\activityname}[1]{
    \title{\centering \Huge Risk Assessment: #1}
    \vspace{25pt}
}

% Details about the activity
% \activitydetails
%   {date}
%   {time}
%   {location}
%   {summary}
\newcommand{\activitydetails}[4]{
    \newcommand{\Dactdet}{
        \section*{Activity Details}
        \begin{tabular*}{\linewidth}[c]{p{2cm}p{\linewidth-2cm}}
            \textbf{Date:}      & #1 \\
            \textbf{Time:}      & #2 \\
            \textbf{Location:}  & #3 \\
            \textbf{Summary:}   & #4 \\
        \end{tabular*}
    }
}

% Details about the assessment
% \assessmentdetails
%   {group}
%   {assessor name}
%   {assessor email}
%   {assessment date}
\newcommand{\assessmentdetails}[4]{
    \newcommand{\Dassdet}{
        \section*{Assessment Details}
        \begin{tabular*}{\linewidth}[c]{p{3cm}p{\linewidth-3cm}}
            \textbf{Student group:}     & #1 \\
            \textbf{Assessor name:}     & #2 \\
            \textbf{Assessor email:}    & \email{#3} \\
            \textbf{Assessment date:}   & #4 \\
        \end{tabular*}
    }
}

% List of references
% Reference items of the form
% \item
%   {description}
%   {source}
\newenvironment{referencelist}{
    \newcommand{\Dreflist}{
        \section*{References}
        \textit{Additional documents or other sources of information that were referred to when preparing this risk assessment}

        \renewcommand\item[2]{
        ##1 \textit{##2}
            \medskip}
        {\bigskip}
    }
}

% List of risks
% Risk items of the form
% \item
%   {hazard}
%   {affected parties}
%   {current control measures}
%   {additional control measures}
%   {likelihood level}{impact level}
\newenvironment{risklist}{
    %\newcommand{\Drisklist}{
        \newpage
        \section*{Risks}

        \begin{longtable}{>{\raggedright}p{4cm}%
                          p{5cm}%
                          p{6cm}%
                          p{6cm}%
                          p{2.2cm}}
            \toprule
            Hazard &
            Who may be affected and how &
            Control measures in place &
            Additional control measures &
            Risk level (/9) \\
            \midrule
            \endhead

            \renewcommand{\item}[6]{
                ##1 &
                ##2 &
                \vspace{-6mm}
                \begin{itemize}
                    \setlength{\itemsep}{0pt plus 1pt}
                    ##3
                \end{itemize} &
                \vspace{-6mm}
                \begin{itemize}
                    \setlength{\itemsep}{0pt plus 1pt}
                    ##4
                \end{itemize} &
                \intcalcMul{##5}{##6} \\
            }

            \bottomrule
        \end{longtable}
    %}
}

% A table with spaces for up to three reviewers to sign and add comments

\newcommand{\Drevlist}{
    \newline
    \section*{Review}
    % A table with spaces for up to three reviewers to sign and add comments.
    \begin{tabular}{|p{6cm}|p{10cm}|p{4cm}|p{3cm}|}
        \hline
        Reviewer name/role &
        Comments &
        Signed &
        Date \\
        \hline
        & & & \\[1.5cm]
        \hline
        & & & \\[1.5cm]
        \hline
        & & & \\[1.5cm]
        \hline
    \end{tabular}
}

\newcommand{\Drisktable}{
    \newline
    \section*{Assessment Guidance}

    Each hazard is assigned a number between 1 and 3 indicating the likelihood of the hazard affecting a person:
    \begin{description}
        \item[Low (1):] May only occur in exceptional circumstances
        \item[Medium (2):] Might occur in some circumstances
        \item[High (3):] Will likely occur in most circumstances
    \end{description}

    Similarly, each hazard is assigned a number between 1 and 3 indicating the magnitude of the impact that the hazard would have, if it did occur:
    \begin{description}
        \item[Low (1):] Superficial or minor injury. Can usually be handled by local first aid procedures.
        \item[Medium (2):] Serious injury, possibly resulting in hospitalisation for up to three days. Complete recovery/rehabilitation could take several months.
        \item[High (3):] Major or fatal injury. Requires extensive medical treatment, including at least three days in hospital.
    \end{description}

    The hazard's \emph{risk level} is then calculated to be the likelihood rating multiplied by the impact rating. For example, a hazard that is likely to occur in almost all circumstances but only results in a minor injury would have a likelihood rating of $3$, an impact rating of $1$, and an overall risk level of $3 \times 1 = 3$.

    These guidelines are based on those provided in Union Southampton's risk assessment template.
}


% === Actual layout of file ===

\raggedcolumns
\begin{multicols*}{2}

\assessmentdetails
\Dreflist
\columnbreak
\Dactdet

\end{multicols*}

\risklist
\Drevlist
\Drisktable

% === PDF SETTINGS ===

\hypersetup
    pdftitle={\Dtitle},
    {unicode=true,
    pdfauthor={\Dauthor}}


