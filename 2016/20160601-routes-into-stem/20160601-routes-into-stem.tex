% ================================================
% ============== ASSESSMENT DETAILS ==============
% ================================================

% The name of the student society.
\newcommand{\groupname}{Student Robotics (Southampton)}

% The name and email address of the person preparing this risk assessment.
\newcommand{\assessorname}{Kier Davis}
\newcommand{\assessoremail}{me@kierdavis.com}

% The date on which this risk assessment was created.
\newcommand{\assessmentdate}{May 19, 2016}


% ================================================
% =============== ACTIVITY DETAILS ===============
% ================================================

% The name of the activity (either an event, or "General Activities").
\newcommand{\activityname}{``Routes Into STEM'' Outreach Day}

% The date(s) and time(s) of the activity.
% For "General Activities", these may be "Wednesday afternoons" and
% "17:00--23:00", for example.
\newcommand{\activitydate}{June 1, 2016}
\newcommand{\activitytime}{All day}

% The location the activity takes place in.
\newcommand{\activitylocation}{Room 53/3028 (Mountbatten building)}

% A summary of the activity. It should describe:
%   - who is attending
%   - what tasks will be done
%   - what equipment will be involved
\newcommand{\activitysummary}{
    Between 15 and 30 local schoolchildren in Year 10 are attending a workshop
    organised by the Electronics and Computer Science department.
    The workshop involves a class led by members of Student Robotics Southampton
    and RoboGals Southampton Chapter, who will be referred to as \emph{mentors}.
    The class involves programming small mBed-based robots to perform a simple
    task. The schoolchildren will be working on laptops provided by the
    department. An arena will be constructed by the mentors for the robots to
    operate in.
}


% ================================================
% ================== REFERENCES ==================
% ================================================

% A list of external sources of information that were referred to when preparing
% this document. 
\newcommand{\references}{
    \reference{Guidance from the Health and Safety Executive, including manual
    handling procedures and case studies of risk assessment writing. \\
    \url{http://www.hse.gov.uk/risk/index.htm}}
    
    \reference{H\&S guidance from the Union Southampton website. \\
    \url{https://www.unionsouthampton.org/groups/admin/howto/protection}}
    
    \reference{Risk assessments prepared for events run by Student Robotics in
    2015. \\
    \url{https://github.com/srobo-southampton/risk-assessments/tree/master/old}}
}


% ================================================
% ==================== RISKS =====================
% ================================================

\newcommand{\risks}{
    \risk
        {Electrical equipment (robots, laptop power supplies)}
        {Mentors or children could get electrical shocks or burns from faulty
         equipment.}
        {\item The robots are low power devices running at no more than 6~volts,
         so the chance of electric shock is very low.
         \item The computing equipment is owned by the university and so will
         have electrically tested.}
        {\item The ESO\footnote{Equipment and Safety Officer} will verify that
         all wiring in the robots is sufficiently insulated.}
        {1} % Likelihood (/3)
        {2} % Impact (/3)
    
    \risk
        {Interaction with autonomous robots}
        {Mentors or children could encounter minor injuries if the robots move
         unexpectedly.}
        {\item Robots are only to be tested under supervision.
         \item When robots are switched on, they will be treated as though they
         could become active at any moment.
         \item The robots do not have any moving parts besides the wheels, which
         are well guarded by the robot's chassis.}
        {\item The ESO will verify that the robots do not present any sharp
         edges. If any are found, they will be covered with thick tape to reduce
         the chance and severity of injury they could cause.}
        {1} % Likelihood (/3)
        {1} % Impact (/3)
    
    \risk
        {Use of manual tools during arena construction}
        {Mentors could experience minor injury as a result of an accident or
         through improper use of tools.}
        {\item Any work requiring use of manual tools will be carried out by
         someone experienced with DIY tasks.
         \item Care will be taken with sharp tools to ensure that minimal injury
         results in the event of an accident.}
        {\item No further action required.}
        {2} % Likelihood (/3)
        {1} % Impact (/3)
    
    \risk
        {Manual handling of heavy objects}
        {Mentors could experience minor injury or back pains resulting from
        improper lifting methods.}
        {\item The HSE manual handling guidelines%
         \footnote{\url{http://www.hse.gov.uk/pubns/indg143.pdf}}
         are to be followed for all tasks involving heavy lifting.}
        {\item No further action required.}
        {1} % Likelihood (/3)
        {2} % Impact (/3)
    
    \risk
        {Obstacles on the floor, such as bags or trailing cables}
        {Children or mentors may suffer minor injury as a result of tripping.}
        {\item Cables (primarily the laptop power supplies) will be routed
         underneath desks wherever possible.
         \item The children will be instructed to keep all bags tucked underneath
         desks.}
        {\item No further action required.}
        {2} % Likelihood (/3)
        {1} % Impact (/3)
}


\NeedsTeXFormat{LaTeX2e}
\ProvidesClass{riskassessment}

% Risk Assessments are landscape and A4
\LoadClass[a4paper,landscape]{article}

% Add packages
\RequirePackage{titlesec,array,calc,footnote}
\RequirePackage{hyperref,intcalc,longtable}
\RequirePackage{multicol}
\RequirePackage[margin=2cm]{geometry}

% Allow footnotes inside tables
\makesavenoteenv{tabular*}

% === CONFIGURE PAGE LAYOUT ===
\setlength{\parskip}{5pt}       % 5pt spacing between paragraphs
\setlength{\parindent}{0pt}     % No paragraph indentation
\setlength{\columnsep}{1.5cm}   % 1.5cm gutter between text columns


% === DEFINE COMMANDS ===

% Show an email address
\newcommand{\email}[1]{\href{mailto:#1}{#1}}

% Set the activity name as a title
\newcommand{\activityname}[1]{
    \title{\centering \Huge Risk Assessment: #1}
    \vspace{25pt}
}

% Details about the activity
% \activitydetails
%   {date}
%   {time}
%   {location}
%   {summary}
\newcommand{\activitydetails}[4]{
    \newcommand{\Dactdet}{
        \section*{Activity Details}
        \begin{tabular*}{\linewidth}[c]{p{2cm}p{\linewidth-2cm}}
            \textbf{Date:}      & #1 \\
            \textbf{Time:}      & #2 \\
            \textbf{Location:}  & #3 \\
            \textbf{Summary:}   & #4 \\
        \end{tabular*}
    }
}

% Details about the assessment
% \assessmentdetails
%   {group}
%   {assessor name}
%   {assessor email}
%   {assessment date}
\newcommand{\assessmentdetails}[4]{
    \newcommand{\Dassdet}{
        \section*{Assessment Details}
        \begin{tabular*}{\linewidth}[c]{p{3cm}p{\linewidth-3cm}}
            \textbf{Student group:}     & #1 \\
            \textbf{Assessor name:}     & #2 \\
            \textbf{Assessor email:}    & \email{#3} \\
            \textbf{Assessment date:}   & #4 \\
        \end{tabular*}
    }
}

% List of references
% Reference items of the form
% \item
%   {description}
%   {source}
\newenvironment{referencelist}{
    \newcommand{\Dreflist}{
        \section*{References}
        \textit{Additional documents or other sources of information that were referred to when preparing this risk assessment}

        \renewcommand\item[2]{
        ##1 \textit{##2}
            \medskip}
        {\bigskip}
    }
}

% List of risks
% Risk items of the form
% \item
%   {hazard}
%   {affected parties}
%   {current control measures}
%   {additional control measures}
%   {likelihood level}{impact level}
\newenvironment{risklist}{
    %\newcommand{\Drisklist}{
        \newpage
        \section*{Risks}

        \begin{longtable}{>{\raggedright}p{4cm}%
                          p{5cm}%
                          p{6cm}%
                          p{6cm}%
                          p{2.2cm}}
            \toprule
            Hazard &
            Who may be affected and how &
            Control measures in place &
            Additional control measures &
            Risk level (/9) \\
            \midrule
            \endhead

            \renewcommand{\item}[6]{
                ##1 &
                ##2 &
                \vspace{-6mm}
                \begin{itemize}
                    \setlength{\itemsep}{0pt plus 1pt}
                    ##3
                \end{itemize} &
                \vspace{-6mm}
                \begin{itemize}
                    \setlength{\itemsep}{0pt plus 1pt}
                    ##4
                \end{itemize} &
                \intcalcMul{##5}{##6} \\
            }

            \bottomrule
        \end{longtable}
    %}
}

% A table with spaces for up to three reviewers to sign and add comments

\newcommand{\Drevlist}{
    \newline
    \section*{Review}
    % A table with spaces for up to three reviewers to sign and add comments.
    \begin{tabular}{|p{6cm}|p{10cm}|p{4cm}|p{3cm}|}
        \hline
        Reviewer name/role &
        Comments &
        Signed &
        Date \\
        \hline
        & & & \\[1.5cm]
        \hline
        & & & \\[1.5cm]
        \hline
        & & & \\[1.5cm]
        \hline
    \end{tabular}
}

\newcommand{\Drisktable}{
    \newline
    \section*{Assessment Guidance}

    Each hazard is assigned a number between 1 and 3 indicating the likelihood of the hazard affecting a person:
    \begin{description}
        \item[Low (1):] May only occur in exceptional circumstances
        \item[Medium (2):] Might occur in some circumstances
        \item[High (3):] Will likely occur in most circumstances
    \end{description}

    Similarly, each hazard is assigned a number between 1 and 3 indicating the magnitude of the impact that the hazard would have, if it did occur:
    \begin{description}
        \item[Low (1):] Superficial or minor injury. Can usually be handled by local first aid procedures.
        \item[Medium (2):] Serious injury, possibly resulting in hospitalisation for up to three days. Complete recovery/rehabilitation could take several months.
        \item[High (3):] Major or fatal injury. Requires extensive medical treatment, including at least three days in hospital.
    \end{description}

    The hazard's \emph{risk level} is then calculated to be the likelihood rating multiplied by the impact rating. For example, a hazard that is likely to occur in almost all circumstances but only results in a minor injury would have a likelihood rating of $3$, an impact rating of $1$, and an overall risk level of $3 \times 1 = 3$.

    These guidelines are based on those provided in Union Southampton's risk assessment template.
}


% === Actual layout of file ===

\raggedcolumns
\begin{multicols*}{2}

\assessmentdetails
\Dreflist
\columnbreak
\Dactdet

\end{multicols*}

\risklist
\Drevlist
\Drisktable

% === PDF SETTINGS ===

\hypersetup
    pdftitle={\Dtitle},
    {unicode=true,
    pdfauthor={\Dauthor}}

